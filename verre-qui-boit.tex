\documentclass{grand-jeu}

\titre{Le verre qui boit}
\categorie{Débrouillard}

\begin{document}

\begin{liste-materiel}
\materiel{Assiette}
\materiel{Plusieurs bougies type chauffe plat}
\materiel{Un verre à moutarde}
\materiel{Un briquet}
\materiel{Un jerrican d’eau}
\materiel{Une pièce de monnaie}
\end{liste-materiel}

\begin{regles}
A faire avec 1 ou 2 équipes

Les membres de l’équipe doivent réussir à sortir une pièce dans une assiette contenant de l’eau sans se mouiller les doigts.

Solution (à ne pas communiquer) : Pour y arriver il faut allumer la bougie, renverser le verre dessus. Lorsque l’air contenu dans le verre aura disparu pour cause de combustion, l’eau de l’assiette sera aspirée dans le verre. Il suffit alors de récupérer la pièce de monnaie.

L’équipe qui a réussi à récupérer l’outil gagne un pois chiche.
\end{regles}

\begin{imaginaire}
Les jeunes sont des chimistes qui doivent récupérer une pépite sans toucher le liquide autour, qui est toxique.
\end{imaginaire}

\begin{moments-stop}
\end{moments-stop}

\end{document}
