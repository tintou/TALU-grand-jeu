\documentclass{grand-jeu}

\titre[66]{La course en cartons}
\categorie{Solidaire}

\begin{document}

\begin{liste-materiel}
\materiel{Autant de morceaux de cartons que de participants}
\end{liste-materiel}

\begin{regles}
Les équipes doivent arriver le plus rapidement possible à un endroit donné, en respectant un parcours défini.

Les joueurs se mettent en colonne devant la ligne de départ, avec chacun un carton. Le premier pose son carton à terre, monte dedans. Puis il prend le carton du joueur derrière lui, et le pose en avant. Pendant ce temps, tous les joueurs donnent leur carton au joueur de devant. Le premier joueur monte dans le carton nouvellement posé, et le deuxième monte dans le premier carton. Puis le premier joueur prend le carton du deuxième, etc. Quand tout le monde sera dans un carton, il n’en restera alors plus qu’un à faire passer d’un bout à l’autre de la colonne. Il est évidemment interdit de poser un pied à l’extérieur des cartons. La colonne devra ainsi effectuer un certain parcours défini à l’avance. Si un joueur tombe hors de son carton ou met un pied à l’extérieur, toute l’équipe recommence le parcours depuis la ligne de départ.

Si l'équipe arrive à faire tout le parcours dans le temps imparti, ils gagnent des b-dures rouge.
\end{regles}

\begin{imaginaire}
Les jeunes sont des ingénieurs qui doivent montrer leurs compétences pour explorer de nouvelles planètes avec des moyens limités et en travaillant ensemble.
\end{imaginaire}

\begin{moments-stop}
\end{moments-stop}

\end{document}