\documentclass{grand-jeu}

\titre[8]{Jeu de Kim}
\categorie{Solidaire}

\begin{document}

\begin{liste-materiel}
\materiel{Des objets variés, 20 environ (ustensiles de cuisine, matériel de campisme, de préférence ressemblant à des planètes, comètes, etc, éléments en lien avec l’imaginaire)}
\materiel{Une bâche, un manteau, une nappe pour recouvrir les objets}
\end{liste-materiel}

\begin{regles}
Répartir les objets devant l’animateur de jeu, cachés par un tissu. Les jeunes ont 3 minutes pour élaborer leur stratégie.

Ensuite, on découvre les objets pendant 1 minute, puis on les recouvre, on en retire deux à l’abri des regards et on découvre de nouveau les objets. Les jeunes ont 1 minute pour retrouver les objets manquants.

La partie se joue en 3 manches, s’ils sont victorieux sur plus de 2, ils gagnent des b-dures.

\vspace{0.2cm}

Pour rendre le jeu solidaire:

faire un plateau assez grand et demander l’équipe de citer ce qui était dessus en laissant deux temps : un pour décider de sa stratégie puis un pour observer et mémoriser (ou rendre la chose plus compliquée en ne séparant pas les deux temps) 
\end{regles}

\begin{imaginaire}
Les astronautes doivent retenir la carte de l’espace car leur ordinateur est tombé en panne et il ne peuvent compter que sur eux-mêmes.

Mais les objets célestes bougent.
\end{imaginaire}

\begin{moments-stop}
\end{moments-stop}

\end{document}
