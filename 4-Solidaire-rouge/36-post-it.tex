\documentclass{grand-jeu}

\titre[36]{Jeu du post-it}
\categorie{Solidaire}

\begin{document}

\begin{liste-materiel}
\materiel{Post-it}
\materiel{Stylos}
\materiel{Liste de personnages}
\end{liste-materiel}

\begin{regles}
Un représentant de chaque équipe a un post-it avec le nom d'un personnage écrit sur le front (adapté à l'âge de l'enfant).

Il pose des questions au reste de l'équipe qui doit lui répondre par oui ou par non et doit deviner le nom du personnage inscrit sur son front.

Lorsque le nom d'un personnage a été trouvé, un autre joueur prend un post-it sur son front et l'équipe marque un point.

Le meneur de jeu doit vérifier que les équipes ne trichent pas en aidant les jeunes à deviner.

L’équipe ayant le plus de point à la fin du temps imparti recevra des b-dures rouge .
\end{regles}

\begin{imaginaire}
Bienvenue au planète des "oui"et "non" dans cette nouvelle planète récemment découvert , on a le droit de répondre que par "oui" ou "non" ; gare à toi si tu te trompes .

\end{imaginaire}

\begin{moments-stop}
\end{moments-stop}

\end{document}
