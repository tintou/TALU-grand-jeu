\documentclass{grand-jeu}

\titre[16]{Le cercle}
\categorie{Solidaire}

\begin{document}

\begin{liste-materiel}
\materiel{Papiers avec des parties du corps (attention à en éviter certaines…)}
\materiel{Sac pour la pioche}
\end{liste-materiel}

\begin{regles}
Chaque participant pioche dans un sac ou chapeau un papier sur lequel est indiqué une partie du corps (pied, genoux, front, talon…).

Il faut que l’équipe forme un cercle mais chaque participant n’a le droit de toucher ses 2 voisins qu’avec la partie du corps indiquée sur le papier.

De plus, les participants ne doivent pas directement se toucher : un papier doit être maintenu entre eux.

L’équipe qui aura réussi recevra des b-dures.
\end{regles}

\begin{imaginaire}
Par accident des fioles contenant des produits toxiques sont tombés et ont infecté les scientifiques sur des partie différentes de leur corps.

Pour récupérer les remèdes, ils doivent réussir à reconstituer un cercle mais attention, il faut surtout que les parties des corps infectées se touchent pour être guéries.
\end{imaginaire}

\begin{moments-stop}
\end{moments-stop}

\end{document}
