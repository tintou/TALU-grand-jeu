\documentclass{grand-jeu}

\titre{Jeu de Kim (Liste)}
\categorie{Débrouillard/Solidaire}

\begin{document}

\begin{liste-materiel}
\materiel{Des objets variés, 20 environ (ustensiles de cuisine, matériel de campisme, de préférence ressemblant à des planètes, comètes, etc, éléments en lien avec l’imaginaire)}
\materiel{Une bâche, un manteau, une nappe pour recouvrir les objets}
\materiel{Une liste papier avec tous les objets, ou un papier et un crayon}
\end{liste-materiel}

\begin{regles}
Répartir les objets devant l’animateur de jeu, cachés par un tissu. Les jeunes ont 3 minutes pour élaborer leur stratégie.

Ensuite, on découvre les objets pendant 1min, puis on les recouvre et les jeunes ont une minute pour citer tous les objets.

S’ils ne réussissent pas, on découvre de nouveau tous les objets et ils font une nouvelle tentative. S’ils réussissent, ils gagnent des b-dures.
\end{regles}

\begin{imaginaire}
Les astronautes doivent retenir la carte de l’espace car leur ordinateur est tombé en panne et il ne peuvent compter que sur eux-mêmes.
\end{imaginaire}

\begin{moments-stop}
\end{moments-stop}

\end{document}
