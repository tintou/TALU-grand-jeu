\documentclass{grand-jeu}

\titre[58]{Basket à ficelle}
\categorie{Débrouillard}

\begin{document}

\begin{liste-materiel}
\materiel{2 ficelles}
\materiel{1 ballon}
\materiel{1 bassine, corbeille}
\end{liste-materiel}

\begin{regles}
En fonction de son âge, le jeune prend place sur une ligne de jeux située plus ou moins loin de l'objectif. Il prend les ficelles dans ses mains. Le meneur de jeu tient les ficelles de l'autre côté, il ne bouge pas ses mains.

Le ballon est déposé sur les ficelles près des mains du joueur.

En levant ou descendant les ficelles, le joueur fait progresser le ballon vers la bassine. Lorsqu'il s'estime au-dessus du panier, il écarte les ficelles dans le but de faire tomber le ballon dans le panier. Si le ballon tombe dans la bassine, le joueur fait gagner un point à son équipe.

On alterne un joueur de chaque équipe.

L'équipe ayant le plus de points à  la fin gagne des b-dures bleu.
 
\end{regles}

\begin{imaginaire}
Les astronautes doivent montrer qu'ils sont habiles et prêts à travailler ensemble pour mener à bien leur mission.  
\end{imaginaire}

\begin{moments-stop}
\end{moments-stop}

\end{document}