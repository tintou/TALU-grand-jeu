\documentclass{grand-jeu}

\titre[22]{La tour d'écrous}
\categorie{Débrouillard}

\begin{document}

\begin{liste-materiel}
\materiel{Écrous}
\materiel{2 bâton/ tiges (diamètre inférieur à celui des écrous) }
\materiel{Plateau/ Zone plate}
\end{liste-materiel}

\begin{regles}
Le maître du jeu et les candidats disposent d'un bâton sur lesquels sont enfilés des écrous. Les duellistes, à l'aide du bâton, doivent chacun leur tour empiler au centre du plateau les écrous sur la tranche. Le premier qui fait tomber la tour d'écrous perd le duel.

S’ils réussissent, ils gagnent des b-dures bleu.
\end{regles}

\begin{imaginaire}
Encore une fois la précision et l'adresse de nos très chers compagnons est mise à l'épreuve.
Pour qu'ils puissent s'en sortir; ils doivent être encore plus soudé et solidaire entre eux.
\end{imaginaire}

\begin{moments-stop}
\end{moments-stop}

\end{document}
