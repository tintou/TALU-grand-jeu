\documentclass{grand-jeu}

\titre[37]{Le 21}
\categorie{Rayonnement}

\begin{document}

\begin{regles}
Deux équipes s’affrontent.

\vspace{0.2cm}
\emph{But du jeu :}

Il faut compter sans se tromper jusqu'à 21. Cependant, au fur et à mesure du jeu les chiffres seront remplacés par des mots. 

\vspace{0.2cm}
\emph{Déroulé :}

Les joueurs des deux équipes s'assoient en cercle en alternant un de chaque équipe. Ils comptent jusqu'à 21 chacun leur tour, dans le sens des aiguilles d’une montre. La personne qui arrive à 21 a le droit de changer un chiffre par un mot. 

On recompte ensuite et la personne qui tombera sur le chiffre qui a été changé ne devra pas citer le chiffre mais le mot correspondant. 

A chaque fois que l'on arrive à 21 on change un nouveau chiffre par un mot. 

Si un jeune se trompe il est éliminé.

Le jeune qui reste le dernier en course gagne des b-dures violet pour son équipe.
\end{regles}

\begin{imaginaire}
Les jeunes doivent apprendre à communiquer de manière différente pour mieux travailler tous ensemble.
\end{imaginaire}

\begin{moments-stop}
\end{moments-stop}

\end{document}