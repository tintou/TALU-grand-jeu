\documentclass{grand-jeu}

\titre[49]{Le saut cumulatif}
\categorie{Rayonnement}

\begin{document}

\begin{liste-materiel}
\materiel{De quoi définir les lignes de départ et d'arrivée: bâtons}
\end{liste-materiel}

\begin{regles}
Les joueurs sautent chacun à leur tour comme au saut en longueur mais en additionnant au fur et à mesure tous les sauts réalisés. 

Un joueur saute puis le joueur suivant saute en partant du point d'arrivée du premier. 

Le but est de dépasser de faire le saut cumulé le plus long (il faut le même nombre de jeunes dans chaque équipe). 

L'équipe qui a fait le plus long saut gagne des b-dures violet .
\end{regles}

\begin{imaginaire}
Les astronautes doivent s'entraîner pour pouvoir explorer différentes planètes. Il faut qu'ils développent leur force et leur agilité tout en travaillant en équipe. 
\end{imaginaire}

\begin{moments-stop}
\end{moments-stop}

\end{document}