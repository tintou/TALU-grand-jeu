\documentclass{grand-jeu}

\titre[23]{Mission: Impossible}
\categorie{Rayonnement}

\begin{document}

\begin{regles}
Les joueurs d'une équipe sont face aux joueurs de l'autre équipe (1 contre 1). 

Les joueurs imposent un rythme par l'élocution des chiffres « 0,0,7 », et chaque joueur se calque sur ce rythme pour agir simultanément. Les deux joueurs face à face doivent déclarer l'action effectuée en même temps, au moment du « 7 ». 

\vspace{0.2cm}
Il y a 3 actions possibles: 
\begin{itemize}
  \item Recharger : les mains sur les tempes (permet de récupérer une munition), dire « recharge ».
  \item Se protéger: les bras en croix devant sa poitrine (permet de ne pas mourir en cas de tir), dire « ouf ». 
  \item Tirer: mains en pistolet vers l'adversaire, dire « pan » (impossible si aucune munition n'est disponible, tue l'adversaire s'il recharge).
\end{itemize}

Dans tous les cas, il faut commencer en rechargeant. Car on ne peut pas tirer et donc se protéger est inutile. Chaque « pistolet » ne peut contenir que 2 balles. L’équipe qui a la dernière personne vivante a gagné.

S’ils réussissent, ils gagnent des b-dures.
\end{regles}

\begin{imaginaire}
a modifier

Les astronautes doivent retenir la carte de l’espace car leur ordinateur est tombé en panne et il ne peuvent compter que sur eux-mêmes.
\end{imaginaire}

\begin{moments-stop}
\end{moments-stop}

\end{document}