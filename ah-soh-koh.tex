\documentclass{grand-jeu}

\titre{Ah Soh Koh}
\categorie{Solidaire}

\begin{document}

\begin{regles}
Se mettre en cercle pour faire passer le Ch'i (énergie de la vie)

Avant de commencer :
\begin{itemize}
    \item{Demandez au groupe de se mettre en cercle. Expliquez qu'une «énergie» sera dirigée autour du cercle avec trois mots / actions différents: «Ah», «Soh» et «Koh».}
    \item{La première passe commence par un «Ah» accompagné d'une main au-dessus de la tête avec les doigts pointés sur la personne de chaque côté. Passez le mot et le mouvement autour du cercle dans la même direction.}
    \item{Passez 'Soh' autour de vous en passant une main sur le ventre dans la direction opposée à celle où 'Ah' était juste passé.}
    \item{Un «Koh» est fait en pointant avec les deux paumes ensemble à n'importe qui dans le cercle, qui passe ensuite un "Ah" à quelqu'un d'autre autour du cercle. Assurez-vous qu'il y a un contact visuel avec la personne qui reçoit le 'Koh'.}
    \item{Les actions doivent toujours aller dans le même ordre: Ah, Soh, puis Koh.}
    \item{Une fois que tout le monde est familier avec les mots et les mouvements, commencez la partie.}
\end{itemize}
Installer :
\begin{itemize}
    \item{Désigner un espace de jeu suffisamment grand pour que le groupe forme un cercle}
\end{itemize}
Comment jouer :
\begin{itemize}
    \item{Le leader commence par un « Ah » et le passe à la personne soit à sa gauche ou à sa droite.}
    \item{C'est à la personne qui le reçoit quelle direction il faut passer, avec un 'Soh'.}
    \item{La prochaine personne doit 'Koh', l'envoyer à quelqu'un à travers le cercle.}
    \item{'Ah' et 'Soh' peuvent être envoyés dans les deux sens.}
    \item{Si une personne hésite, elle doit contourner l'extérieur du cercle et essayer de tromper les autres dans le cercle, en disant les mots Ah, Soh, Koh dans les oreilles des gens (mais ne touchez aucun joueur).}
\end{itemize}
\end{regles}

\begin{moments-stop}
\end{moments-stop}

\end{document}
