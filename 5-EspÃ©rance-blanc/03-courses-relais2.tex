\documentclass{grand-jeu}

\titre[3]{Courses de relais 2}
\categorie{Espérance}

\begin{document}

\begin{liste-materiel}
\materiel{De quoi matérialiser le parcours :}
\materiel{Quelques plots}
\materiel{Balles de ping-pong}
\materiel{Cuillères à soupe}
\end{liste-materiel}


\begin{regles}
Les deux équipes doivent réaliser deux parcours similaires en parallèle (slalom, enjamber le banc, tourner plusieurs fois autour d'un plot, courir, etc) sans faire tomber la balle de la cuillère et sans la toucher. 

Si la balle tombe, le joueur doit recommencer le parcours au début.

Il faut qu'il y aie le même nombre de joueurs qui passent par équipe, si besoin un joueur peut passer plusieurs fois.

L'équipe qui a fini le relais en premier gagne des b-dures. 
\end{regles}

\begin{imaginaire}
\end{imaginaire}

\begin{moments-stop}
\end{moments-stop}

\end{document}