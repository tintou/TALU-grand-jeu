\documentclass{grand-jeu}

\titre[45]{Les îlots}
\categorie{Espérance}

\begin{document}

\begin{liste-materiel}
\materiel{Carrés de tissus/cartons pour faire les îlots}
\end{liste-materiel}

\begin{regles}
Dispersez les îlots sur le sol, entre deux repères tracés, situés aux deux extrémités du terrain de jeu. Les îlots devront être de la taille d'un pied, et devront être suffisamment éloignés les uns des autres pour que les jeunes aient une grande enjambée à faire pour passer de l'un à l'autre.

La première équipe s'aligne au départ alors que l'autre se regroupe de l'autre côté du parcours (assurez-vous qu'ils n'empiètent pas sur le parcours lui-même). 
Au signal, les jeunes de la première équipe doivent traverser la pièce en marchant sur les îlots pendant que l'équipe adverse, sans les toucher, fait tout pour essayer de les troubler de façon à leur faire perdre l'équilibre. Celui qui touche le sol en dehors des îlots doit retourner au départ et recommencer. 

Chaque équipe a trois minutes durant lesquelles le plus grand nombre d'enfants tentent de terminer le parcours et de passer la ligne d'arrivée. Lorsque le temps s'est écoulé, les équipes changent de rôles.

L'équipe qui a fait passer la ligne d'arrivée au plus grand nombre de ses membres gagne des b-dures
\end{regles}

\begin{imaginaire}
Les astronautes doivent s'entraîner pour pouvoir explorer différentes planètes.Il faut qu'ils développent leur agilité et leur rapidité. 
\end{imaginaire}

\begin{moments-stop}
\end{moments-stop}

\end{document}


