\documentclass{grand-jeu}

\titre[62]{Jeu du papier cadeau}
\categorie{Espérance}

\begin{document}

\begin{liste-materiel}
\materiel{Deux gants de cuisine}
\materiel{Plusieurs cadeaux très bien emballés : Plein couches de papier, de scotch etc.}
\materiel{Un dé à 6 faces}
\end{liste-materiel}

\begin{regles}
Deux équipes s'affrontent. Les joueurs se placent assis en cercle, équipes alternées un joueur sur deux.

Le plus jeune joueur commence et mets les deux gants de cuisine, puis il essaie de déballer l'object avec ses gants.

Le joueur à sa gauche lance les dés jusqu'à ce qu'il arrive à faire 6. Il remplace alors le joueur qui déballe le cadeau et enfile les gants.

Le joueur à sa gauche se précipite ainsi à jeter le dé et le jeu continue jusqu'à ce que l'objet soit déballé. Plus un joueur met de temps à faire 6 et plus son voisin de droite a de temps pour essayer de déballer le cadeau.

L'équipe qui réussi à déballer le cadeau et le brandis au-dessus de sa tête gagne le jeu.
\end{regles}

\begin{imaginaire}
Les jeunes doivent apprendre à travailler ensemble et à être complémentaire les uns des autres pour mener à bien leur mission et rétablir l'harmonie entre eux.
\end{imaginaire}

\begin{moments-stop}
\end{moments-stop}

\end{document}