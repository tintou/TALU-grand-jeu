\documentclass{grand-jeu}

\titre[19]{L'aquarium}
\categorie{Espérance}

\begin{document}

\begin{liste-materiel}
\materiel{Grand récipient (saladier, casserole…)}
\materiel{Verre}
\materiel{Pièces, billes ou autres.}
\materiel{Eau}
\end{liste-materiel}

\begin{regles}
Au centre de la table trône un aquarium (ou récipient) rempli d'eau et au milieu duquel flotte un verre.

Les membres de l'équipe et le maître du jeu disposent chacun d'une dizaine de pièces (ou autre), de tailles identiques ou non. À tour de rôle, ils vont devoir mettre une de leurs pièces dans le verre sans le faire couler. 

Celui qui fait couler le verre dans l'aquarium perd le duel.

S’ils réussissent, les joueurs gagnent des b-dures blanc. 
\end{regles}

\begin{imaginaire}
Le père fouras (dans Fort Boyard), vous met au défis de faire mieux que sont plus grand élèves (l'animateur du jeu) , pour le battre il faudra faire preuve de précision et faire attention aux conséquences de leurs gestes! 
\end{imaginaire}

\begin{moments-stop}
\end{moments-stop}

\end{document}

