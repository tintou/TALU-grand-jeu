\documentclass{grand-jeu}

\titre{Chemin/Labyrinthe en aveugle}
\categorie{Énergie}

\begin{document}

\begin{liste-materiel}
\materiel{Foulards des jeunes pour cacher les yeux.}
\materiel{Plots pour définir les obstacles}
\end{liste-materiel}

\begin{regles}
Le jeu peut se jouer à une ou deux équipes. Soit il s’agit d’une course, soit d’une course contre la montre.

Un ou deux parcours doivent être définis, avec des virages, des éléments à contourner etc.

Les jeunes d’une équipe se mettent en file indienne avec les yeux bandés, sauf le dernier. Il doit guider son équipe : tapoter l’épaule gauche : à gauche, tapoter l’épaule droite : à droite, taper sur les deux épaules en même temps : avancer tout droit. Les joueurs n’ont pas le droit de parler, tout le guidage doit se faire avec les mains.

L’équipe la plus rapide gagne, ou si elle arrive à faire le parcours dans le temps imparti. L'équipe gagne des b-dures.
\end{regles}

\begin{imaginaire}
Dans l’espace, il fait noir et seules quelques personnes ont la capacité de guider tous les autres, ce sont les pilotes et ils doivent guider tous les autres à bon port.
\end{imaginaire}

\begin{moments-stop}
\end{moments-stop}

\end{document}
