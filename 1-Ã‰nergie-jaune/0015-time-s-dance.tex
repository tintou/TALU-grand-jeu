\documentclass{grand-jeu}

\titre{Courses de relais}
\categorie{Énergie}

\begin{document}

\begin{liste-materiel}
\materiel{Papiers avec des mouvements de danse (2 fois les mêmes) + les mêmes papiers mais avec un mot au dos (2 fois les mêmes)}
\materiel{Chronomètre}
\materiel{Éventuellement : de la musique}
\end{liste-materiel}

\begin{regles}
Comme le time’s up, deux équipes s'affrontent.

Sur des papiers, il y a un bonhomme qui fait un mouvement (simple) de danse particulier.

Un membre de l’équipe tire un papier et doit refaire la chorégraphie dessinée dessus.

Quand il commence à danser, un autre membre de son équipe doit courir chercher dans un tas de papier posés plus loin, le papier avec la danse qui correspond, retenir le mot qui se trouve au dos et revenir le dire à son danseur.

Si ce n’est pas le bon papier, il doit retourner chercher le bon. S’il a juste, un autre membre de l’équipe se met à danser, et un autre part trouver le bon papier.

Il est possible d’ajouter un fond de musique pour animer encore un peu plus le jeu.

L’équipe qui a réussi à récupérer l’outil gagne des b-dures.
\end{regles}

\begin{imaginaire}
Les compétences et capacités des astronautes sont soumis à rude épreuve pour pouvoir sans sortir ils devront rester solidaire eux.
\end{imaginaire}

\begin{moments-stop}
\end{moments-stop}

\end{document}
