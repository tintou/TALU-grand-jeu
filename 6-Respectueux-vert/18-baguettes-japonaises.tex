\documentclass{grand-jeu}

\titre[18]{Baguette japonaise}
\categorie{Respectueux}

\begin{document}

\begin{liste-materiel}
\materiel{Baguettes chinoises/japonaises }
\materiel{Petits et/ou plus grands objets (billes, noix, chamallows, balle de tennis, figurines, bouteilles…)}
\materiel{De quoi représenter la zone où les éléments doivent être amenés.}
\end{liste-materiel}

\begin{regles}
Il faut attraper des objets avec des baguettes japonaises/ chinoises et les transporter jusqu’à un autre endroit. Il est possible de faire une variante plus difficile, ou 2 joueurs tiennent les baguettes de manière horizontale et doivent déplacer les objets ensemble.

S’ils réussissent, ils gagnent des b-dures vert.
\end{regles}

\begin{imaginaire}
Les jeunes sont des astronautes en sortie dans l’espace depuis leur navette, ils doivent récupérer un objet qui se trouvent dans un endroit très étroit et leur gant ne leur permettent pas trop de faire des manipulations,ils doivent redoubler d'imagination pour récupérer ses objets.


\end{imaginaire}

\begin{moments-stop}
\end{moments-stop}

\end{document}
