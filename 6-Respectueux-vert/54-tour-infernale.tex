\documentclass{grand-jeu}

\titre[54]{La tour infernale}
\categorie{Respectueux}

\begin{document}

\begin{liste-materiel}
\materiel{Une surface plane (bassine, retournée, carton, bois)}
\materiel{jeu de construction en vois type kapla}
\end{liste-materiel}

\begin{regles}
Le placement des pièces sur la tour se fait perpendiculairement aux pièces situées à l'étage inférieur. Chaque étage est composé de 3 pièces.

Les joueurs retirent progressivement les pièces pour les replacer à son sommet jusqu'à ce qu'elle finisse par perdre l'équilibre. On alterne un jeune de chaque équipe.

Le gagnant est le dernier joueur à avoir déplacé une pièce de bois sans faire tomber la tour, son équipe gagne des b-dures vert.
\end{regles}

\begin{imaginaire}
Les astronautes doivent travailler leur agilité tout en travaillant en équipe pour finaliser leur entraînement. 
\end{imaginaire}

\begin{moments-stop}
\end{moments-stop}

\end{document}