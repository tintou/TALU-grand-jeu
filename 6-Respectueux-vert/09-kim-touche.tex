\documentclass{grand-jeu}

\titre[8]{Jeu de Kim}
\categorie{Respectueux}

\begin{document}

\begin{liste-materiel}
\materiel{Des objets variés, 20 environ (ustensiles de cuisine, matériel de campisme, de préférence ressemblant à des planètes, comètes, etc, éléments en lien avec l’imaginaire)}
\materiel{Une bâche, un manteau, une nappe pour recouvrir les objets}
\end{liste-materiel}

\begin{regles}
Répartir les objets devant l’animateur de jeu, cachés par un tissu. 

Les jeunes doivent identifier tous les objets en un minimum de temps, en passant les mains sous la couverture.
Deux jeunes peuvent chercher en même temps.

Attention de ne pas soulever la couverture! Si c'est compliqué, on peut bander les yeux des joueurs.

Si les joueurs ont réussi à trouver tous les objets dans le temps imparti, ils gagnent des b-dures vert.
\end{regles}

\begin{imaginaire}
Les astronautes doivent identifier des objets dans le noir car la planète où ils sont est dépourvue de lumière. 
\end{imaginaire}

\begin{moments-stop}
\end{moments-stop}

\end{document}