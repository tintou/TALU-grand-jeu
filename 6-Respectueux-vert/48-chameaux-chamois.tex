\documentclass{grand-jeu}

\titre[48]{Chameaux-Chamois}
\categorie{Respectueux}

\begin{document}

\begin{liste-materiel}
\materiel{Bâtons, cordes pour matérialiser les camps}
\end{liste-materiel}

\begin{regles}
Disposer trois lignes à une dizaine de mètres d'écart. 

Une des équipes est celle des chameaux, l’autre celle des chamois. Chacune d’entre elles a un camp (une des deux lignes du bout chacune).

Tout le monde se place sur la ligne du milieu. Le meneur de jeu prononce alors soit « Chameau » soit « Chamois ». S’il prononce « chamois », les chamois courent vers leur camp poursuivis par les chameaux. Les chamois attrapés avant d’être arrivés au camp sont éliminés et on recommence. C’est bien sûr l’inverse qui se passe lorsque le meneur prononce « chameau » : les chameaux courent vers leur camp poursuivi par les chamois.

Pour que ce soit plus amusant, le meneur du jeu raconte une histoire tortueuse en créant le suspense sur le mot qui sera prononcé. 

L'équipe qui a le plus de joueurs encore en lice à la fin du temps imparti gagne des b-dures vert.
\end{regles}

\begin{imaginaire}
Les astronautes doivent s'entraîner pour pouvoir explorer différentes planètes.Il faut qu'ils développent leurs réflexes. 
\end{imaginaire}

\begin{moments-stop}
\end{moments-stop}

\end{document}

