\documentclass{grand-jeu}

\titre[4]{Courses de relais}
\categorie{Respectueux}

\begin{document}

\begin{regles}
2 équipes s’affrontent (Attention, des jeunes de même tranche d'âge) :

Les joueurs des deux équipes doivent se mettre en file indienne et se faire passer la balle en arrière l’un au dessus de la tête et le suivant entre les jambes puis au dessus de la tête entre jambes et ainsi de suite jusqu’au dernier.

\vspace{0.2cm}

Si jeu réussit, noter le point et passer au suivant :

Le premier joueur prend la balle, fait un slalom le plus rapidement possible et la ramène pour la passer au deuxième joueur et ainsi de suite. L’équipe la plus rapide gagne.

Les joueurs s’attachent deux pieds ensemble (course à trois pattes) et doivent faire un parcours le pus rapidement possible.

\vspace{0.2cm}

Bonus si ex-aequo : Course de vitesse : 10 m en courant, 10 m à cloche pieds, 10 m en sauts de grenouille, 10 m en ayant un ballon entre ses genoux, 10 m en marche arrière.

Si l’équipe a réussi à faire passer la balle jusqu’au bout sans la faire tomber, le jeu est réussi.

L’équipe qui a gagne remporte des b-dures.
\end{regles}

\begin{imaginaire}
Avec balle : Les habitants des planètes doivent transporter des minerais fragiles qui ne doivent pas toucher le sol, sinon ils se cassent.

Sans balle : Les jeunes doivent montrer leur habileté à se déplacer rapidement dans différentes conditions qui correspondent à différentes planètes.
\end{imaginaire}

\begin{moments-stop}
\end{moments-stop}

\end{document}
